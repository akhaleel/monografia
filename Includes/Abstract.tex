% Resumo em l�ngua estrangeira (em ingl�s Abstract, em espanhol Resumen, em franc�s R�sum�)
\begin{center}
	{\Large{\textbf{Data Processing with Apache Spark and Storm: A vision on e-participation in Brazilian State Capitals}}}
\end{center}

\vspace{1cm}

\begin{flushright}
	Author: Felipe Cordeiro Alves Dias\\
	Advisor: N�lio Alessandro Azevedo Cacho, Doctor
\end{flushright}

\vspace{1cm}

\begin{center}
	\Large{\textsc{\textbf{Abstract}}}
\end{center}

\noindent In the context of Smart Cities, one of the challenges is the processing of large volumes of data in continuous expansion given the steady increase of people and objects connected to the Internet. In this scenario, it is possible for citizens to participate virtually of issues addressed by its respective local government, which is essential for the development of Smart Cities and known as e-Participation. Because of this, this monograph analyzes the existing level of e-Participation in Brazilian state capitals, questioning already ranked as Smart, using for that two of the main tools for data processing: Spark Apache and Apache Storm.

\noindent\textit{Keywords}: e-Participation, Data Processing, Smart Cities.