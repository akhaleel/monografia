% Resumo em l�ngua estrangeira (em ingl�s Abstract, em espanhol Resumen, em franc�s R�sum�)
\begin{center}
	{\Large{\textbf{Data Processing with Apache Spark and Storm: A vision on e-participation in Brazilian State Capitals}}}
\end{center}

\vspace{1cm}

\begin{flushright}
	Author: Felipe Cordeiro Alves Dias\\
	Advisor: N�lio Alessandro Azevedo Cacho, Doctor
\end{flushright}

\vspace{1cm}

\begin{center}
	\Large{\textsc{\textbf{Abstract}}}
\end{center}

\noindent In the Smart Citie's contexts, one of the challenges is to process the large volume of data, in continuous expansion of data, due to the constant increase of people and objects connected to the internet. In this scenario, it is possible for the citizens to virtually participate of questions addressed by your respective local government, which is essential for the development of Smart Cities and known as eParticipation. For this reason this thesis aims to collect, through \textit{tweets} (Social Network Twitter posts), metrics related to e-Participation of the Brazilian State Capitals, mapping them and questioning the already classified as Smart Cities, utilizing for this, two of the main Data Processing tools: Apache Spark and Apache Storm.

\noindent\textit{Keywords}: e-Participation, Data Processing, Smart Cities.