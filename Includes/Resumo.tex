% Resumo em l�ngua vern�cula
\begin{center}
	{\Large{\textbf{Processamento de dados com Apache Spark e Storm: Uma vis�o sobre e-Participa��o nas capitais dos estados brasileiros}}}
\end{center}

\vspace{1cm}

\begin{flushright}
	Autor: Felipe Cordeiro Alves Dias\\
	Orientador: Doutor N�lio Alessandro Azevedo Cacho
\end{flushright}

\vspace{1cm}

\begin{center}
	\Large{\textsc{\textbf{Resumo}}}
\end{center}

\noindent No contexto de Cidades Inteligentes, um dos desafios � o processamento de grande volumes de dados, em cont�nua expans�o dado o aumento constante de pessoas e objetos conectados � Internet. Nesse cen�rio, � poss�vel que os cidad�os participem, virtualmente, das quest�es abordadas por seu respectivo governo local, algo essencial para o desenvolvimento das Cidades Inteligentes e conhecido como e-Participa��o. Devido a isso, essa monografia se prop�e a coletar, atrav�s de \textit{tweets} (publica��es da Rede Social Twitter), m�tricas relacionadas a e-Participa��o das capitais dos estados brasileiros, mapeando-as e questionando as j� classificadas como Cidades Inteligentes, utilizando para isso duas das principais ferramentas para processamento de dados: Apache Spark e Apache Storm.

\noindent\textit{Palavras-chave}: Cidades Inteligentes, e-Participa��o, Processamento de dados.